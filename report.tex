\documentclass[a4paper,11pt,openany,oneside,article]{memoir}
\usepackage{amsfonts}
\usepackage[fleqn]{amsmath}
\usepackage{amssymb}
\usepackage{amsthm}
\usepackage[english,dutch]{babel}
\usepackage{graphicx}
\usepackage[colorlinks,linkcolor=black,citecolor=black,filecolor=black,urlcolor=black,hypertexnames=false,naturalnames=false]{hyperref}
\usepackage{mathtools}
\usepackage{xspace}
\selectlanguage{dutch}

% must come after hyperref
\usepackage{cleveref}

% fonts
\usepackage[T1]{fontenc}
\usepackage[sc]{mathpazo}
\usepackage{tgpagella}
\usepackage{fixltx2e}
\usepackage{microtype}

\theoremstyle{definition}
\newtheorem{theorem}{Stelling}
\newtheorem{lemma}[theorem]{Lemma}
\newtheorem{corollary}[theorem]{Gevolg}
\newtheorem{definition}[theorem]{Definitie}
\newtheorem{example}[theorem]{Voorbeeld}
\newtheorem{remark}[theorem]{Opmerking}
\crefname{theorem}{Stelling}{Stellingen}
\crefname{lemma}{Lemma}{Lemma's}
\crefname{definition}{Definitie}{Definities}
\crefname{example}{Voorbeeld}{Voorbeelden}
\crefname{remark}{Opmerking}{Opmerkingen}

\newtheorem{exercise}{Exercise}
\newenvironment{solution}[1][\textbf{Solution}\hspace{1em}]{#1}{\hfill\qed}

% http://tug.org/pracjourn/2006-3/robertson/robertson.pdf
\newcommand\foreign[1]{\emph{#1}}
\newcommand\eg{\foreign{e.g.}}
\newcommand\Eg{\foreign{E.g.}}
\newcommand\ie{\foreign{i.e.}}
\newcommand\Ie{\foreign{I.e.}}
\newcommand\vs{\foreign{vs.~}}
\newcommand\cf{\foreign{cf.\@}}
\newcommand\resp{\foreign{resp.\xspace}}
\newcommand\etc{\foreign{etc.}}
\newcommand\etal{\foreign{et al}}

\newcommand\ShfSpace{\mathrm{L}}
\newcommand\Shf{\mathrm{\Gamma}}

\DeclareMathOperator\identity{id}
\DeclareMathOperator\Image{Im}


\begin{document}
\title{\textls[20]{Project Schoventheorie \\ Wat is algebra\"ische topologie?}}
\author{
  \protect\includegraphics[width=2cm]{universiteitantwerpen-eps-converted-to.pdf} \\[3em]
  Pieter Belmans
}
\maketitle

\tableofcontents

\section{Voorwoord}
Zoals gebruikelijk voor een vak bij prof.~dr.~Van~Oystaeyen schrijft een student een werk(je) waarin hij iets vertelt dat vaagweg met de cursus te maken heeft. Dit keer is het niet anders. Voor het vak Schoventheorie\footnote{De laatste keer dat het in zijn huidige vorm gegeven wordt ontdekte ik onlangs, dit is bijgevolg de laatste keer dat zo'n werk geschreven wordt.} heb ik gekozen om een tweeledig werk te schrijven.

Enerzijds wil ik de twee (of vier, als je de co- apart telt) grote pijlers van de algebra\"ische topologie defini\"eren: homotopie, homologie en cohomologie. Een bachelorstudent hoort deze vaak, zonder echt te weten wat deze nu exact zijn. Om mezelf (en mijn medestudenten, moesten deze ge\"interesseerd zijn in wat ik hier schrijf) voor eens en voor altijd duidelijk te maken wat deze termen betekenen geef ik een korte inleiding.

Daarna geef ik een introductie tot bettigetallen, een concrete invulling van de algebra\"ische topologie en het verband van deze getallen met schoofcohomologie. Ook hier moet ik (noodgedwongen) op de vlakte blijven: noch het formaat, noch mijn kennis staan me toe om zeer diep te gaan.

Desalniettemin hoop ik mezelf (en de eventuele lezer) iets bij te leren en wens ik professor~Van~Oystaeyen te bedanken voor de interessante gesprekken (voor zover het geen bezoek aan het orakel was).


\section{Een duik in de algebra\"ische topologie}
Topologen zijn zeer ge\"interesseerd in de classificatie van topologische ruimten. Met behulp van zogenaamde \emph{topologische invarianten} proberen ze om een onderscheid te kunnen maken tussen ruimten: dat twee ruimten homeomorf zijn is enkel aan te tonen door een expliciet homeomorfisme te construeren. Dit is vaak niet mogelijk. Daarom wordt de situatie omgekeerd: door aan te tonen dat twee ruimten \emph{verschillen} voor een bepaalde invariant zijn ze alvast niet homeomorf.

In het vak Analytische topologie\footnote{Eerste semester in de tweede bachelor, door prof.~dr.~Lowen} worden verschillende van deze topologische invarianten besproken:
\begin{description}
  \item[separatie-eigenschappen] Zoals daar zijn hausdorff, regulier, normaal beschrijven deze invarianten in welke mate we in staat zijn om punten of deelruimten van elkaar te scheiden. 

  \item[aftelbaarheidsvoorwaarden] Overdekkingen of (lokale) omgevingenbasissen kunnen aftelbare delen bevatten, zodat het voldoende is om slechts naar dit aftelbaar deel te kijken om alsnog algemene uitspraken te kunnen doen.
    
    Separabiliteit behelst dan weer een aftelbaar dicht deel binnen de ruimte. Aangezien rijen aftelbare structuren zijn is het voor uitspraken die hiermee in verband staan handig om aan bepaalde voorwaarden te voldoen.
    
  \item[samenhang] Hierbij wordt er gekeken of de ruimte al dan niet uiteenvalt in deelruimten met bepaalde eigenschappen. Ook het gedrag van paden (en veralgemeningen hiervan) wordt hier bestudeerd. Dit is het soort topologische invarianten die we zullen veralgemenen in wat volgt.

  \item[compactheid] Overdekkingen van ruimtes kunnen ook aan bepaalde aftelbaarheidsvoorwaarden voldoen, maar vaak zullen we hier zelfs \emph{eindige} deeloverdekkingen eisen. 

  \item[metriseerbaarheid] De analytische topologie is ontstaan als een veralgemening van de studie der metrische ruimten, bijgevolg is het ook interessant om te kunnen zeggen wanneer een ruimte homeomorf is met een metrische ruimte en dus dezelfde eigenschappen heeft. Op basis van deze metriek kunnen dan zeer sterke eigenschappen van de ruimte gebruikt worden: al de besproken klassen hebben ruwweg een ordening van zwak naar sterk en metriseerbare ruimten voldaan een sterke invarianten.
\end{description}

Al deze invarianten hebben ook een lokale varianten, waarbij er enkel wordt gekeken naar een arbitraire omgeving van een punt. Voor interessante voorbeelden die deze topologische invarianten van elkaar scheiden verwijzen we naar~\cite{counterexamples-in-topology}.
\vspace{1em}

Al deze invarianten vinden hun oorsprong binnen de analytische topologie. Maar door een goede vertaling van de topologie naar algebra te maken zullen we in staat zijn om een nieuwe klasse van topologische invarianten aan te boren. Zo komen we tot de \emph{algebra\"ische topologie}.

\subsection{Homotopie}
Hierbij zijn we ge\"interesseerd in de continue deformaties van continue functies tussen twee topologische ruimten naar elkaar. Als zo'n deformatie (die we een \emph{homotopie} zullen noemen) mogelijk is, zijn de functies in een bepaald aspect gelijk aan elkaar. Formeel:

\begin{definition}
  Een \emph{homotopie} voor~$f,g\colon X\to Y$ met~$f,g$ continu en~$X,Y$ topologische ruimten is een continue afbeelding~$H\colon X\times[0,1]\to Y$ zodat~$H(x,0)=f(x)$ en~$H(x,1)=g(x)$.
\end{definition}

Algebra\"ische topologie heeft geen alleenrecht op de notie van homotopie, maar veel concepten binnen de algebra\"ische topologie zullen invariant zijn voor homotopie. Het maakt dus niet uit welke functie we kiezen zolang ze maar homotoop zijn aan elkaar. In~\cref{subsection:homology-and-cohomology} zien we invarianten die gelijk zijn voor homotope ruimten, maar eerst bespreken we de homotopiegroepen.

In de cursus Analytische topologie zagen we reeds de notie van fundamentaalgroep. Hiermee werd de verzameling van alle klassen van lussen in een basispunt~$x_0$ van een topologische ruimte~$X$ bedoeld. Op basis hiervan kunnen we zeggen of lussen al dan niet samentrekbaar in de ruimte. Formeel:

\begin{definition}
  Uit de verzameling van \emph{lussen}~$f\colon[0,1]\to X$ met~$f(0)=f(1)=x_0$ delen we de equivalentierelatie die ge\"induceerd wordt door homotopie weg. Hierop zetten we vervolgens het product van klassen door~$[f]\ast [g]$ te defini\"eren voor representanten~$f$ en~$g$ als
  \begin{equation}
    f\ast g=
      \begin{cases}
        f(2t) & t\in\left[0,\frac{1}{2}\right] \\
        g(2t-1) & t\in\left[\frac{1}{2},1\right]
      \end{cases}
  \end{equation}
  en het is aan te tonen dat~$[f\ast g]$ onafhankelijk is van de keuze van representanten.

  Het bekomen object is \emph{de fundamentaalgroep~$\fundamental(X,x_0)$} van~$X$ in het punt~$x_0$.
\end{definition}

We associ\"eren we dus aan een topologische ruimte (of algemener: een object binnen een categorie) een algebra\"isch object (of algemener: een veelal simpeler object in een andere categorie) dat nog steeds voldoende informatie bevat over het oorspronkelijke object om hier zinnige uitspraken over te kunnen doen. In de taal van categorie\"en: een functor. In het geval van de fundamentaalgroep is dit~$\fundamental\colon\TopP\to\Grp$.

\begin{example}
  \begin{enumerate}
    \item De fundamentaalgroep~$\fundamental(\mathbb{R}^n,x_0)$ met op~$\mathbb{R}^n$ de euclidische topologie is voor elke~$n$ de triviale groep: elke lus kan zonder enig probleem worden teruggetrokken tot het punt~$x_0$: er zijn geen obstakels.

    \item De fundamentaalgroep~$\fundamental(\sphere^1,x_0)$ van de cirkel~$\sphere^1$ is echter~$\mathbb{Z}$. Een lus op de cirkel komt overeen met de notie van windingsgetallen: het aantal keer dat deze rondgaat ligt vast onder homotopie.

      Er kan bewezen worden dat de fundamentaalgroep van het product van topologische ruimten het product is van de fundamentaalgroepen van de componenten. Bijgevolg is~$\fundamental(\torus^n,x_0)=\prod_{i=1}^n\mathbb{Z}$.

    \item Voor~$\sphere^n$ met~$n\geq 2$ kunnen we echter aantonen dat lussen terug samentrekbaar worden: voor de bol~$\sphere^2$ is dit intu\"itief duidelijk omdat er geen obstakels in de weg zitten van zo'n contractie. Bijgevolg is~$\fundamental(\sphere^n,x_0)=\left\{ e \right\}$ voor~$n\geq 2$.

    \item De fles van Klein heeft in tegenstelling tot de vorige voorbeelden geen abelse fundamentaalgroep:~$\fundamental(K,x_0)=\Dih_{\infty}$, de symmetriegroep van~$\mathbb{Z}$ voortgebracht door~$a\colon n\mapsto n+1$ en~$b\colon n\mapsto -n$. Net zoals de symmetriegroepen van de een regelmatige~$n$\nobreakdash-hoek is deze groep niet-abels. 
  \end{enumerate}
\end{example}

We merken reeds een probleem bij~$\fundamental(\mathbb{R}^n,x_0)$ en~$\fundamental(\sphere^n,x_0)$ voor~$n\geq 2$: deze ruimten zijn duidelijk verschillend van elkaar, maar onze topologische invariant is niet in staat om hier onderscheid tussen te maken\footnote{Nu zijn er wel meerdere invarianten die hiervoor falen: beide ruimten zijn bijvoorbeeld hausdorff. Maar compactheid daarentegen gaat enkel op voor~$\sphere^n$.}. We zullen dus een veralgemening invoeren.

In plaats van lussen (die homotoop zijn met~$\sphere^1$) in een punt~$x_0$ zullen we naar sferen~$\sphere^n$ kijken en zien hoe deze zich gedragen. Bijgevolg is~$\homotopyGroup_n(X,x_0)$ de verzameling van homotopieklassen van afbeeldingen~$f\colon\sphere^n\to X$ die een basispunt op de sfeer naar~$x_0$ sturen.

Maar we kunnen ook volgende definitie\footnote{Waarom niet de meer intu\"itieve versie met sferen? Wel, voor~$f\ast g$ met~$f,g\colon\sphere^n\to X$ is de groepsbewerking~$\Phi\circ h$ waar~$\Phi\colon\sphere^n\to\sphere^n\wedge\sphere^n/\sim$ waarbij~$\sim$ twee grote cirkels op de factoren identificeert en~$\wedge$ de wedgesom is, het coproduct in~$\TopP$ dat gedefinieerd is door~$(X,x_0)\wedge(Y,y_0)/\left\{ (x_0,y_0) \right\}
$ en~$h$ is gedefinieerd als~$h\colon\sphere^n\wedge\sphere^n\to X$ zodat~$h$ op de eerste sfeer~$f$ is en op de tweede~$h$. Net \emph{daarom} dat we de minder intu\"itieve maar gemakkelijkere versie zullen defini\"eren.} invoeren:

\begin{definition}
  De~$n$\nobreakdash-de homotopiegroep~$\homotopyGroup_n(X,x_0)$ is de verzameling homotopieklassen voor afbeeldingen~$f\colon[0,1]^n\to X$ waarbij de groepsbewerking gedefinieerd is als
  \begin{equation}
    f\ast g=
    \begin{cases}
      f(2t_1,t_2,\ldots,t_n) & t_1\in\left[ 0,\frac{1}{2} \right] \\
      g(2t_1-1,t_2,\ldots,t_n) & t_1\in\left[ \frac{1}{2},1 \right].
    \end{cases}
  \end{equation}
\end{definition}

\begin{example}
  Het is vooral populair om homotopiegroepen op sferen te bepalen, voor andere objecten is het vaak te moeilijk om deze uit te rekenen\footnote{Daarom dat we homologie zullen invoeren.}.

  Op basis van de stelling van Hurewicz die homologie en homotopie met elkaar in verband brengt is het mogelijk om aan te tonen dat~$\homotopyGroup_n(\sphere^n,x_0)=\mathbb{Z}$. Verder geldt dat~$\homotopyGroup_n(\sphere^{n+k},x_0)=\left\{ e \right\}$ voor~$k\geq 1$.

  Het zijn dus de gevallen~$\homotopyGroup_{n+k}(\sphere^n,x_0)$ voor~$k\geq 1$ die interessant zullen blijken. Bijvoorbeeld~$\homotopyGroup_3(\sphere^2,x_0)=\mathbb{Z}$, op basis van een afbeelding genaamd de Hopf{}fibratie. Zoals de fundamentaalgroep van de cirkel de hele groep werd voortgebracht door een enkele lus, die een aantal keer (in een bepaalde richting) doorlopen kon worden, brengt de Hopf{}fibratie heel~$\homotopyGroup_3(\sphere^2,x_0)$ voort.
  
  Dit resultaat suggereert dat homotopiegroepen steeds~$\mathbb{Z}$ zijn. Maar er kan aangetoond worden dat~$\homotopyGroup_4(\sphere^2,x_0)=\homotopyGroup_4(\sphere^3,x_0)=\mathbb{Z}/2\mathbb{Z}$, een eindige groep zowaar. De enige regelmaat die er is bestaat uit het feit dat de hogere homotopiegroepen steeds abels zijn. Maar dat~$\homotopyGroup_{14}(\sphere^4,x_0)=\mathbb{Z}/120\mathbb{Z}\times\mathbb{Z}/12\mathbb{Z}\times\mathbb{Z}/2\mathbb{Z}$ is zeker geen intu\"itief resultaat. In~\cite{composition-methods} is een overzicht te vinden van de reeds bepaalde homotopiegroepen op~$\sphere^n$.
\end{example}

\begin{remark}
  Er bestaat ook de notie van \emph{cohomotopie}. De afbeeldingen worden simpelweg omgedraaid, dus de $n$\nobreakdash-de cohomotopiegroep~$\homotopyGroup^n(X,x_0)$ bestaat uit de klassen van~$f\colon X\to\sphere^n$. Ze zijn echter veel minder onderzocht dan homotopie en (co)homologie, daarom dat in de inleiding slechts~$3$~pijlers gesuggereerd werden\footnote{Ik had er nog nooit van gehoord totdat ik aan dit werkje begon.}. Wel interessant is dat~$\homotopyGroup^1(X,x_0)\cong\homologyGroup^1(X)$ de eerste cohomologiegroep.
\end{remark}

\subsection{Homologie en cohomologie}\label{subsection:homology-and-cohomology}
Waar homotopie met een gepunte topologische ruimte een groep associeerde, zal (co)homologie een topologische ruimte in verband brengen met een reeks van abelse groepen of modulen. TODO

\begin{definition}
  Een \emph{ketencomplex} is een reeks~$(A_n,d_n)_{n\in\mathbb{Z}}$ van abelse groepen (of modulen) die aan elkaar worden geplakt met behulp van de homomorfismen~$d_n\colon A_n\to A_{n-1}$ (\emph{randoperator} genaamd\footnote{TODO afgrijselijke vertaling verbeteren}) zodat de samenstelling~$d_n\circ d_{n+1}=0$ voor elke~$n$.
\end{definition}



\section{Bettigetallen en schoofcohomologie}\label{section:betti-numbers}
We zagen dat de abelse groep~$\homologyGroup_k(X)$ ons informatie verschaft over het aantal gaten in een topologische ruimte. We defini\"eren nu simpelweg:
\begin{definition}
  Het~\emph{$k$\nobreakdash-de bettigetal~$\betti_k(X)$} is de rang van de groep~$\homologyGroup_k(X)$ of equivalent de dimensie van de vectorruimte~$\homologyGroup_k(X;\mathbb{Q})$.
\end{definition}

Nu lijkt deze definitie niet veel nieuwe inzichten te verschaffen, maar oorspronkelijk was homologie \emph{niet} de studie van rijen van abelse groepen. De eerste incarnatie (dat wil zeggen: door Poincar\'e, rond~1900) waren net deze bettigetallen, het inzicht dat er groepen mee geassocieerd waren heeft~$20$~jaar op zich laten wachten. Bovendien bieden deze getallen een intu\"itieve interpretatie ($\betti_k(X)$ stemt overeen met het aantal~$k$\nobreakdash-dimensionale gaten in~$X$) van homologie die we verder zullen toelichten in~\cref{example:betti-numbers}.

We associ\"eren met elke topologische ruimte dus een element van~$(\mathbb{N}\cup\left\{ +\infty \right\})^\mathbb{N}$, een rij van natuurlijke getallen of~$+\infty$. Interessant is dat deze bettigetallen in veel interessante gevallen vanaf een bepaald moment~$0$~worden. Dit hangt samen met een zekere vorm van eindigdimensionaliteit: bijvoorbeeld voor compacte manifolds is dit het geval.

Met een rij getallen kunnen we een een \emph{voortbrengende functie} associ\"eren door het~$n$\nobreakdash-de getal als co\"effici\"ent voor de monoom~$x^n$ te nemen. Dit leidt ons tot:

\begin{definition}
  De \emph{poincar\'eveelterm}~$\poincare(X)$ van een topologische ruimte~$X$ is de veelterm~$\sum_{n=0}^{+\infty}\betti_n(X) x^n$.
\end{definition}

\begin{remark}
  De poincar\'eveelterm is enkel een veelterm voor topologische ruimten waarbij de reeks bettigetallen slechts voor eindig veel waarden~$0$~is. Anders verkrijgen we een reeks.
\end{remark}

In~\cref{subsection:homology-and-cohomology} kwamen geen voorbeelden aan bod\footnote{TODO klopt dit nog steeds?}, daar brengen we nu verandering in.
\begin{example}
  \label{example:betti-numbers}
  \begin{enumerate}
    \item Voor de cirkel~$\sphere^1$ wordt de reeks van bettigetallen gegeven door~$1,1,0,0,0,\ldots$ dus de poincar\'eveelterm is simpelweg~$1+x$.
      
      Nu kunnen we de eerste bettigetallen interpreteren als volgt:
      \begin{description}
        \item[$\betti_0(X)$] geeft het aantal samenhangscomponenten van de ruimte;
        \item[$\betti_1(X)$] geeft het aantal tweedimensionale gaten;
        \item[$\betti_2(X)$] geeft het aantal driedimensionale gaten of leegten.
      \end{description}

      We zien dat dit van toepassing is op~$\sphere^1$: er is~$1$~samenhangscomponent en de cirkel omschrijft de open schijf, die het tweedimensionale gat vormt.

    \item De $n$\nobreakdash-torus~$\torus^n$ wordt gerealiseerd als product van~$n$~cirkels~$\sphere^1$ en het is aan te tonen dat de poincar\'eveelterm van een product het product van de poincar\'eveeltermen der factoren is. We vinden dus
      \begin{equation}
        \poincare(\torus^n)=\prod_{i=1}^n\poincare(\sphere^1)=(1+x)^n=\sum_{k=0}^n\binom{n}{k}x^k,
      \end{equation}
      zodat de bettigetallen gegeven worden door
      \begin{equation}
        1,\binom{n}{1},\ldots,\binom{n}{n},0,0,0,\ldots
      \end{equation}

      We kunnen deze omgekeerde beweging nu helemaal te voltrekken door te zeggen dat de rang van~$\homologyGroup_k(\torus^n)$ gegeven wordt door~$\binom{n}{k}$. Meer nog, er geldt dat:~$\homologyGroup_k(\torus^n)\cong\mathbb{Z}^{\binom{n}{k}}$.

      Voor de intu\"itieve interpretatie in het geval~$n=2$:~$\betti_0(\torus^2)$ vertelt ons dat er~$1$~samenhangscomponent is, wat zeer duidelijk is. Verder zijn er twee~$2$\nobreakdash-dimensionale gaten: deze zijn respectievelijk het gat in de torus en de holte in het binnenste van de torus. Tot slot is er~\'e\'en~$3$\nobreakdash-dimensionaal gat: het gat middenin de torus, maar dit keer in~$3$~dimensies opgevat.

  \end{enumerate}
\end{example}




\nocite{*}
\bibliographystyle{alpha}
\bibliography{bibliography}

\end{document}
