\documentclass[a4paper,11pt]{article}
\usepackage{amsfonts}
\usepackage[fleqn]{amsmath}
\usepackage{amssymb}
\usepackage{amsthm}
\usepackage[english,dutch]{babel}
\usepackage{graphicx}
\usepackage[colorlinks,linkcolor=black,citecolor=black,filecolor=black,urlcolor=black,hypertexnames=false,naturalnames=false]{hyperref}
\usepackage{mathtools}
\usepackage{xspace}
\selectlanguage{dutch}

% must come after hyperref
\usepackage{cleveref}

% fonts
\usepackage[T1]{fontenc}
\usepackage[sc]{mathpazo}
\usepackage{tgpagella}
\usepackage{fixltx2e}
\usepackage{microtype}

\theoremstyle{definition}
\newtheorem{theorem}{Stelling}
\newtheorem{lemma}[theorem]{Lemma}
\newtheorem{corollary}[theorem]{Gevolg}
\newtheorem{definition}[theorem]{Definitie}
\newtheorem{example}[theorem]{Voorbeeld}
\newtheorem{remark}[theorem]{Opmerking}
\crefname{theorem}{Stelling}{Stellingen}
\crefname{lemma}{Lemma}{Lemma's}
\crefname{definition}{Definitie}{Definities}
\crefname{example}{Voorbeeld}{Voorbeelden}
\crefname{remark}{Opmerking}{Opmerkingen}

\newtheorem{exercise}{Exercise}
\newenvironment{solution}[1][\textbf{Solution}\hspace{1em}]{#1}{\hfill\qed}

% http://tug.org/pracjourn/2006-3/robertson/robertson.pdf
\newcommand\foreign[1]{\emph{#1}}
\newcommand\eg{\foreign{e.g.}}
\newcommand\Eg{\foreign{E.g.}}
\newcommand\ie{\foreign{i.e.}}
\newcommand\Ie{\foreign{I.e.}}
\newcommand\vs{\foreign{vs.~}}
\newcommand\cf{\foreign{cf.\@}}
\newcommand\resp{\foreign{resp.\xspace}}
\newcommand\etc{\foreign{etc.}}
\newcommand\etal{\foreign{et al}}

\newcommand\ShfSpace{\mathrm{L}}
\newcommand\Shf{\mathrm{\Gamma}}

\DeclareMathOperator\identity{id}
\DeclareMathOperator\Image{Im}


\begin{document}
\title{\textls[20]{Project Schoventheorie \\ Wat is algebra\"ische topologie?}}
\author{
  \includegraphics[width=2cm]{universiteitantwerpen-eps-converted-to.pdf} \\[3em]
  Pieter Belmans
}
\maketitle

\tableofcontents

\section{Voorwoord}
Zoals gebruikelijk voor een vak bij prof.~dr.~Van~Oystaeyen schrijft een student een werk(je) waarin hij iets vertelt dat vaagweg met de cursus te maken heeft. Dit keer is het niet anders. Voor het vak Schoventheorie\footnote{De laatste keer dat het in zijn huidige vorm gegeven wordt ontdekte ik onlangs, dit is bijgevolg de laatste keer dat zo'n werk geschreven wordt.} heb ik gekozen om een tweeledig werk te schrijven.

Enerzijds wil ik de twee (of drie) grote pijlers van de algebra\"ische topologie defini\"eren: homotopie, homologie en cohomologie. Een bachelorstudent hoort deze vaak, zonder echt te weten wat deze nu exact zijn. Om mezelf (en mijn medestudenten, moesten deze ge\"interesseerd zijn in wat ik hier schrijf) voor eens en voor altijd duidelijk te maken wat deze termen betekenen geef ik een korte inleiding.

Daarna geef ik een introductie tot Bettigetallen, een concrete invulling van de algebra\"ische topologie en het verband van deze getallen met schoofcohomologie. Ook hier moet ik (noodgedwongen) op de vlakte blijven: noch het formaat, noch mijn kennis staan me toe om zeer diep te gaan.

Desalniettemin hoop ik mezelf (en de eventuele lezer) iets bij te leren en wens ik professor~Van~Oystaeyen te bedanken voor de interessante gesprekken (voor zover het geen bezoek aan het orakel was).


\section{Een duik in de algebra\"ische topologie}
\subsection{Homotopie}

\subsection{Homologie}

\subsection{Cohomologie}


\section{Bettigetallen en schoofcohomologie}

\bibliographystyle{alpha}
\bibliography{bibliography}

\end{document}
