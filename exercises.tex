\documentclass[a4paper,11pt]{article}
\usepackage[fleqn]{amsmath}
\usepackage{amsthm}
\usepackage[english,dutch]{babel}
\selectlanguage{dutch}
\usepackage[fixlanguage]{babelbib}
\selectbiblanguage{dutch}
\usepackage{enumitem}
\usepackage{float}
\usepackage{graphicx}
\usepackage[colorlinks,linkcolor=black,citecolor=black,filecolor=black,urlcolor=black,hypertexnames=false,naturalnames=false]{hyperref}
\usepackage{mathtools}
\usepackage{rotating}
\usepackage{tikz}
\usepackage{xspace}

% must come after hyperref
\usepackage[capitalise,dutch]{cleveref}
\crefname{figure}{Figuur}{Figuren}
\crefname{section}{Hoofdstuk}{Hoofdstukken}
\crefname{subsection}{Hoofdstuk}{Hoofdstukken}

% colors
\definecolor{uablue}{RGB}{0,61,100}
\definecolor{uared}{RGB}{126,0,47}
\definecolor{vividbrown}{RGB}{215,154,70}
\definecolor{uagreen}{RGB}{0,126,17}

\usetikzlibrary{matrix,arrows}

% memoir tweaking
\renewcommand{\thesection}{\arabic{section}}

% babelbib sucks
\declarebtxcommands{dutch}{%
  \def\btxnumeralshort#1{%
    \btxnumeralenglish{dutch}{#1}}%
  \def\btxnumerallong#1{%
    \ifnumber{#1}{%
      \ifcase#1 nulde\or eerste\or tweede\or derde\or
        vierde\or vijfde\or zesde\or zevende\or
        achtste\or negende\or tiende\else
        \btxnumeralenglish{dutch}{#1}%
      \fi}{#1}}%
}

% fonts
\usepackage[T1]{fontenc}
\usepackage[charter]{mathdesign}
\usepackage[scaled]{beramono,berasans}
\usepackage{microtype}
\frenchspacing

\addtolength\headheight{12pt}
\addtolength\parskip{.7ex}
\setlength\parindent{0cm}

\relpenalty=10000
\binoppenalty=10000

\theoremstyle{definition}
\newtheorem{theorem}{Stelling}
\newtheorem{lemma}[theorem]{Lemma}
\newtheorem{corollary}[theorem]{Gevolg}
\newtheorem{definition}[theorem]{Definitie}
\newtheorem{example}[theorem]{Voorbeeld}
\newtheorem{remark}[theorem]{Opmerking}
\crefname{theorem}{Stelling}{Stellingen}
\crefname{lemma}{Lemma}{Lemma's}
\crefname{definition}{Definitie}{Definities}
\crefname{example}{Voorbeeld}{Voorbeelden}
\crefname{remark}{Opmerking}{Opmerkingen}

\newtheorem*{exercise}{Exercise}
\newenvironment{solution}[1][\textbf{Solution}\hspace{1em}]{#1}{\hfill\qed}

% http://tug.org/pracjourn/2006-3/robertson/robertson.pdf
\newcommand\foreign[1]{\emph{#1}}
\newcommand\eg{\foreign{e.g.}}
\newcommand\Eg{\foreign{E.g.}}
\newcommand\ie{\foreign{i.e.}}
\newcommand\Ie{\foreign{I.e.}}
\newcommand\vs{\foreign{vs.~}}
\newcommand\cf{\foreign{cf.\@}}
\newcommand\resp{\foreign{resp.\xspace}}
\newcommand\etc{\foreign{etc.}}
\newcommand\etal{\foreign{et al}}

\newcommand\betti{\mathrm{b}}
\newcommand\Dih{\mathrm{Dih}}
\newcommand\eulercharacteristic{\chi}
\newcommand\fundamental{\homotopyGroup_1}
\newcommand\Grp{\mathsf{Grp}}
\newcommand\homologyGroup{\mathrm{H}}
\newcommand\homotopyGroup{\pi}
\newcommand\poincare{\mathrm{P}}
\newcommand\PCat{\mathsf{Psh}}
\newcommand\SCat{\mathsf{Sh}}
\newcommand\sections{\mathrm{\Gamma}}
\newcommand\Sets{\textsf{Sets}}
\newcommand\ShfSpace{\mathrm{L}}
\newcommand\Shf{\mathrm{\Gamma}}
\newcommand\sphere{\mathbb{S}}
\newcommand\TopP{\mathsf{Top}_{\bullet}}
\newcommand\torus{\mathbb{T}}

\DeclareMathOperator\Coker{Coker}
\DeclareMathOperator\Hom{Hom}
\DeclareMathOperator\identity{id}
\DeclareMathOperator\Image{Im}
\DeclareMathOperator\Ker{Ker}
\DeclareMathOperator\Ob{Ob}


\begin{document}
\selectlanguage{english}
\title{\textls[20]{Solutions to Tennison}}
\author{
  Pieter Belmans
}
\maketitle

\tableofcontents

\section{Solutions of Chapter 1}
\begin{enumerate}
  \item Prove that the~$\varinjlim$ of the direct systems of the examples is~$F(\emptyset)$ in each case. Generalise.

    \begin{solution}
      The direct system of a topology corresponds to a complete lattice, hence there is a minimum and a maximum: $\emptyset$ and~$X$. The direct limit now corresponds to `the object that goes to the right of everything' but in this case,~$\emptyset$ is on the right over everything, therefore, the direct limit is given by~$F(\emptyset)$.
    \end{solution}

  \item Prove directly from the definitions that if~$U$ is a direct limit of a direct system~$(U_\alpha)_{\alpha\in\Lambda}$ of sets, then~$U=\bigcup_{\alpha\in\Lambda}\Image(U_\alpha\rightarrow U)$.

    \begin{solution}
      TODO
    \end{solution}

  \item
    \begin{enumerate}
      \item Interpret and prove: a set is the direct limit of its finite subsets.
      
        \begin{solution}
          The construction of the direct limit consists of quotienting the disjoint union. All elements of the set~$U$ are represented in a finite subset (for instance the singletons), so the disjoint union contains all elements. Elements are now identified if they are identified in the power set lattice of~$U$, but as the union of two finite subsets is still finite the identification of an element~$v\in U_\alpha,U_\beta$ occurs in~$U_\gamma=U_\alpha\cup U_\beta$.
        \end{solution}

      \item Interpret and prove: an abelian group is the direct limit of iets finitely generated subgroups.

        \begin{solution}
          Analogously, finitely generated subgroups are represented in the same kind of lattice structure with atoms and finite unions of generators. 
        \end{solution}

      \item Can you obtain~$\mathbb{Z}$ as a direct limit of finite abelian groups?

        \begin{solution}
          TODO
        \end{solution}
    \end{enumerate}

  \item
    \begin{enumerate}
      \item Characterise direct systems of sets with~$\varinjlim=\emptyset$.

        \begin{solution}
          TODO
        \end{solution}

      \item Produce an interesting direct system of abelian groups with~$\varinjlim=\left\{ 0 \right\}$, the trivial group. Characterise such systems.

        \begin{solution}
          TODO
        \end{solution}
    \end{enumerate}

  \item What can you say about the direct limit of a direct system all of whose maps are injective? Surjective?

    \begin{solution}
      All the maps to the direct limit are injective (or surjective). In the injective case the direct limit contains information on~\emph{all} objects of the direct system, in the surjective case the direct limit contains the~\emph{common} information of all objects.
    \end{solution}

  \item For~$n\in\mathbb{N}_0$, let~$C_n(x)$ denote a cyclic group of order~$n$ with generator~$x$. Let~$p\in\mathbb{N}$ be a prime number. Let~$G$ be the direct limit of the following direct system of abelian groups:
    \begin{equation}
      \left\{ 0 \right\}=C_{p^0}(x_0)\rightarrow C_{p^1}(x_1)\rightarrow\ldots\rightarrow C_{p^n}(x_n)\rightarrow C_{p^{n+1}}(x_{n+1})\rightarrow\ldots
    \end{equation}
    (where~$C_{p^n}(x_n)\rightarrow C_{p^{n+1}}(x_{n+1})$ takes~$x_n\mapsto px_{n+1}$). Preferably without resorting to the explicit construction prove:
    \begin{enumerate}
      \item $G$ is infinite, but torsion (\ie~element has finite order).

        \begin{solution}
          Based on the previous exercise we realize that~$G$ is a big structure as every map is injective. As every step in the chain adds a finite number of elements and there are countable steps, we obtain a countable infinite group.

          As the construction of the direct limit consists of summing all objects of the direct system and taking a quotient, we get that every element of the direct sum has finite order (namely the order of the highest term) and the quotient keeps this structure.
        \end{solution}

      \item Every finitely-generated subgroup of~$G$ is finite. Find all of them.

        \begin{solution}
          Every finite set of generators has a maximal index~$n$ such that the term for~$C_{p^n}$ is nonzero but for~$C_{p^{n+1}}$ is zero. The order of the finitely-generated subgroup is now limited by the order~$p^n$.
        \end{solution}
    \end{enumerate}

    Deduce that~$G$ has no proper infinite subgroup, and no maximal proper subgroup. Can either of these situations arise for subspaces of a vector space (using dimension instead of order)? Identify a realisation of~$G$ inside the unit circle~$S^1\subseteq\mathbb{C}$ (under multiplication).

    \begin{solution}
      The construction of the direct sum learns us that there must be an infinite number of nonzero summands, but as every~$C_{p^n}$ is trivially embedded in~$C_{p^m}$ for all~$m>n$ and every such instance is identified we cannot get to a proper infinite subgroup. Analogously there is no maximal proper subgroup.

      The only interesting case occurs in infinite-dimensional vector spaces, the finite-dimensional case is trivial (take the quotient of the space with the field).

      TODO

      The obtained structure is the Pr\"ufer group.
    \end{solution}

  \item Consider the following direct system of abelian groups: fix~$r\in\mathbb{Z}$; for all~$n\in\mathbb{N}$ let~$U_n=\mathbb{Z}$ and for~$n\geq m$ let~$\rho_{m,n}\colon U_m\to U_n$ be multiplication by~$r^{n-m}$. Identify the~$\varinjlim$ as a subring of~$\mathbb{Q}$.

    \begin{solution}
      TODO
    \end{solution}

  \item Interpet and prove: the direct limit of a system of exact sequences is exact.

    \begin{solution}
      TODO
    \end{solution}

  \item The notions of target and direct limit can be formulated without the restriction (a) of Definitions~3.1 and~3.11. What difference does this make to the constructions? Find a system of abelian groups (in this generalised sense) with direct limit~$A\oplus B$ without having this abelian group appear in the system. Justify Remark~3.21.

    \begin{solution}
      TODO
    \end{solution}

  \item Formulate the dual notions of inverse system and inverse limit~$\varprojlim$ (reverse the arrows).

    \begin{solution}
      TODO
    \end{solution}
    
    Find inverse systems:
    \begin{enumerate}
      \item of finite sets whose~$\varprojlim$ is infinite;

        \begin{solution}
          Take~$(\mathbb{N},\leq)$ and define~$A_k=\left\{ 0,\ldots,k \right\}$. We get~$\rho_{i,j}\colon A_j\to A_i$ the projection for~$i\leq j$. The direct limit is the set~$\mathbb{N}$.
        \end{solution}

      \item of finite abelian groups whose~$\varprojlim$ is infinite;

        \begin{solution}
          The ring of $p$\nobreakdash-adic integers~$\mathbb{Z}_p$ is given by~$\varprojlim\mathbb{Z}/p^n\mathbb{Z}$.
        \end{solution}

      \item of abelian groups whose~$\varprojlim$ is~$\mathbb{Z}$ (without~$\mathbb{Z}$ in the system).

        \begin{solution}
          Start from~$\mathbb{Q}$. Let~$p_n$ denote the~$n$th prime number. Now define~$G_k$ to be the subgroup of~$\mathbb{Q}$ in which all primes but~$p_1,\ldots,p_n$ are allowed as factors of the denumerator. The morphisms are the injections. Now~$\mathbb{Z}$ is the inverse limit, it being the subgroup of~$\mathbb{Q}$ in which all prime factors of denumerators are removed.
        \end{solution}
    \end{enumerate}

  \item Verify that if~$(R_\alpha)_{\alpha\in\Lambda}$ is a direct system of abelian groups such that each~$R_\alpha$ is a ring and all the~$\rho_{\alpha,\beta}$ are ring morphisms, then~$\varinjlim R_\alpha$ has a natural ring structure such that all the maps~$R_\beta\to\varinjlim R_\alpha$ are ring morphisms.

    \begin{solution}
      TODO
    \end{solution}

  \item What are the stalks of the presheaf~$P_2$ of~2.E?

    \begin{solution}
      We have~$(P_2)_{x_0}=\mathbb{Z}$ as every open subset~$U$ containing~$x_0$ has~$P_2(U)=\mathbb{Z}$, with~$\rho_{\alpha,\beta}=\identity_{\mathbb{Z}}$ in the direct system.

      For~$(P_2)_{x}$ with~$x\neq x_0$ we get~$\left\{ 0 \right\}$, the trivial group. We have~$|x-x_0|=\epsilon>0$, so for the open set~$B(x,\epsilon)\cap[0,1]$ the presheaf gives~$\left\{ 0 \right\}$, all values of the presheaf from there on are the same trivial group.
    \end{solution}

  \item Construct a topological space~$X$ and a presheaf~$F$ of abelian groups on~$X$ with the properties:
    \begin{enumerate}
      \item for any open~$U\subseteq X$: $F(U)\neq\left\{ 0 \right\}$;
      \item for all~$x\in X$ the stalk~$F_x=\left\{ 0 \right\}$.
    \end{enumerate}
    If you cannot, prove that it is impossible. Compare with~4(b).

    \begin{solution}
      An interesting case with non-abelian groups\footnote{Is this a form of non-abelian sheaf theory?} would be to take~$X=[0,1]$ and assign to an open subset~$U$ the permutation on its elements. The direct limit is now the permutation on a single element, hence the trivial group.

      TODO
    \end{solution}
\end{enumerate}


\section{Solutions to Chapter 2}
\begin{enumerate}
  \item Let~$I=[0,1]\hookrightarrow\mathbb{R}$. Show that there is a unique (up to isomorphism) sheaf~$F$ on~$I$ with stalks: $F_0=F_1=\mathbb{Z}$, $F_x=\left\{ 0 \right\}$ if~$x\in I\setminus\left\{ 0,1 \right\}$.

    \begin{solution}
      TODO
    \end{solution}

    What is~$\Shf(I, F)$?

    \begin{solution}
      It is~$\mathbb{Z}\oplus\mathbb{Z}$, as it is composed of contributions of the stalks in~$0$ and~$1$.
    \end{solution}

    Let~$G$ be the constant sheaf~$\mathbb{Z}$ on~$I$. How many morphisms are there from~$F$ to~$G$? From~$G$ to~$F$?

    \begin{solution}
      TODO
    \end{solution}

  \item Show that the following conditions are equivalent for a topological space~$X$:
    \begin{enumerate}
      \item\label{exercise:2-2-a} $X$ is locally connected (that is, each point has a base of connected neighbourhoods);
      \item\label{exercise:2-2-b} for any set~$A$, the constant sheaf~$A_X$ has~$\Gamma(U,A_X)=\prod_{t\in U'}A$ for~$U$ open in~$A$ where~$U'$ is the set of connected components of~$U$;
      \item\label{exercise:2-2-c} \ref{exercise:2-2-b} holds for~$A=\left\{ 0,1 \right\}$, some set with two elements.
    \end{enumerate}

    \begin{solution}
      \begin{description}
        \item[$\ref{exercise:2-2-a}\Rightarrow\ref{exercise:2-2-b}$] This is implied by the glueing condition for sheaves.
        \item[$\ref{exercise:2-2-b}\Rightarrow\ref{exercise:2-2-c}$] Trivial.
        \item[$\ref{exercise:2-2-c}\Rightarrow\ref{exercise:2-2-a}$] TODO
      \end{description}
    \end{solution}

    When these conditions hold, what are the restriction maps in terms of the representation given in~\ref{exercise:2-2-b}?

    \begin{solution}
      The restrictions are projections on the remaining connected components. For~$V\subseteq U$ open we have the inclusion~$V'\subseteq U'$ of sets of connected components.
    \end{solution}

  \item Let~$F$ be a presheaf on a space~$X$, and let~$V$ be open in~$X$. Then we can define a presheaf~$F|_V$ on~$V$ by the same recipe as~$F$; that is~$(F|_V)(U)=F(U)$ for~$U$ open in~$V$.

    Show that is~$F$ is a sheaf, so is~$F|_V$. Show also that if~$F$ has sheaf space~$p\colon \ShfSpace(F)\to X$, then~$F|_V$ has sheaf space $(p^{-1}(V),p|_{p^{-1}(V)})$. What can you say when~$V$ is not open?

    \begin{solution}
      TODO
    \end{solution}

    Compare~\ref{exercise:2-4} and~\P3.8.

    \begin{solution}
      TODO
    \end{solution}

  \item Let~$F$ be a sheaf on a space~$X$ with sheaf space~$\ShfSpace(F)\overset{p}{\rightarrow} X$, and let~$A$ be a subspace of~$X$. We ca define the set (or abelian group) of sections of~$F$ over~$A$ by
    \begin{align}
      \sections(A,F)&=\sections(A,\Shf(F)) \\
      &=\left\{ \text{sections of the continuous map $p^{-1}(A)\overset{p}{\rightarrow}A$} \right\}.
    \end{align}
    Show that we can define~$\sections(A,F)$ in terms of~$F$ alone as~$\sections(A,F)=\varinjlim\sections(U,F)$ where the direct limit is taken over the set of open subsets~$U$ of~$X$ such that~$U\supseteq A$. (Colloquially, this says that a section of~$F$ over~$A$ extends uniquely into a small neighbourhood of~$A$.)

    \begin{solution}
      TODO
    \end{solution}

  \item Let~$F$ be a sheaf on a space~$X$ and let~$(M_i)_{i\in I}$ be a locally finite covering of~$X$ by closed sets (so that for each~$x\in X$, $\left\{ i\in I\,|\,x\in M_i \right\}$ is finite). In the notation of~\ref{exercise:2-4}, suppose we are given a family~$(s_i)_{i\in I}$ with~$\forall i\in I\colon s_i\in\sections(M_i,F)$ and~$\forall i,j\in I\colon s_i=s_j$ on~$M_i\cap M_j$.

    Show that there is a unique~$s\in\sections(X,F)$ with~$\forall i\in I\colon s=s_i$ on~$M_i$.

    \begin{solution}
      TODO
    \end{solution}

  \item TODO
\end{enumerate}


\section{Solutions to Chapter 3}
\begin{enumerate}
  \item Let~$P$ be the category of pointed sets, whose objects are the pairs~$(A,a)$ with~$a\in A\in\Ob(\Sets)$, and whose morphisms $(A,a)\to(B,b)$ are the maps of sets~$f\colon A\to B$ such that~$f(a)=b$. Show that~$P$ is a category with a zero object, kernels and cokernels, but in which not every epimorphism is a cokernel.

    \begin{solution}
      Every singleton set~$(\left\{ x \right\},x)$ is a zero object. The initial object~$\emptyset$ from~$\Sets$ is now changed to~$(\left( x \right),x)$ as every morphism needs to map the base point. The terminal object~$\left\{ x \right\}$ in~$\Sets$ remains unchanged as it allows for exactly one morphism (which preserves the base point).

      For kernels and cokernels we need a zero morphism, which in this case is the projection on the base point. For~$f\colon(A,a)\to(B,b)$ we have~$\Ker(f)=(f^{-1}(b),a)$ and~$\Coker(f)=(f(A),b)$.

      TODO
    \end{solution}
\end{enumerate}

\bibliographystyle{alpha}
\bibliography{bibliography}

\end{document}
