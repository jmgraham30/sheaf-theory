\usepackage[fleqn]{amsmath}
\usepackage{amsthm}
\usepackage[english,dutch]{babel}
\selectlanguage{dutch}
\usepackage[fixlanguage]{babelbib}
\selectbiblanguage{dutch}
\usepackage{colonequals}
\usepackage{enumitem}
\usepackage{float}
\usepackage{graphicx}
\usepackage[colorlinks,linkcolor=black,citecolor=black,filecolor=black,urlcolor=black,hypertexnames=false,naturalnames=false]{hyperref}
\usepackage{mathtools}
\usepackage{rotating}
\usepackage{tikz}
\usepackage{xspace}

% must come after hyperref
\usepackage[capitalise,dutch]{cleveref}
\crefname{figure}{Figuur}{Figuren}
\crefname{section}{Hoofdstuk}{Hoofdstukken}
\crefname{subsection}{Hoofdstuk}{Hoofdstukken}

% colors
\definecolor{uablue}{RGB}{0,61,100}
\definecolor{uared}{RGB}{126,0,47}
\definecolor{vividbrown}{RGB}{215,154,70}
\definecolor{uagreen}{RGB}{0,126,17}

\usetikzlibrary{matrix,arrows}

% memoir tweaking
\renewcommand{\thesection}{\arabic{section}}

% babelbib sucks
\declarebtxcommands{dutch}{%
  \def\btxnumeralshort#1{%
    \btxnumeralenglish{dutch}{#1}}%
  \def\btxnumerallong#1{%
    \ifnumber{#1}{%
      \ifcase#1 nulde\or eerste\or tweede\or derde\or
        vierde\or vijfde\or zesde\or zevende\or
        achtste\or negende\or tiende\else
        \btxnumeralenglish{dutch}{#1}%
      \fi}{#1}}%
}

% fonts
\usepackage[T1]{fontenc}
\usepackage[charter]{mathdesign}
\usepackage[scaled]{beramono,berasans}
\usepackage{microtype}
\frenchspacing

\addtolength\headheight{12pt}
\addtolength\parskip{.7ex}
\setlength\parindent{0cm}

\relpenalty=10000
\binoppenalty=10000

\theoremstyle{definition}
\newtheorem{theorem}{Stelling}
\newtheorem{lemma}[theorem]{Lemma}
\newtheorem{corollary}[theorem]{Gevolg}
\newtheorem{definition}[theorem]{Definitie}
\newtheorem{example}[theorem]{Voorbeeld}
\newtheorem{remark}[theorem]{Opmerking}
\crefname{theorem}{Stelling}{Stellingen}
\crefname{lemma}{Lemma}{Lemma's}
\crefname{definition}{Definitie}{Definities}
\crefname{example}{Voorbeeld}{Voorbeelden}
\crefname{remark}{Opmerking}{Opmerkingen}

\newtheorem*{exercise}{Exercise}
\newenvironment{solution}[1][\textbf{Solution}\hspace{1em}]{#1}{\hfill\qed}

% http://tug.org/pracjourn/2006-3/robertson/robertson.pdf
\newcommand\foreign[1]{\emph{#1}}
\newcommand\eg{\foreign{e.g.}}
\newcommand\Eg{\foreign{E.g.}}
\newcommand\ie{\foreign{i.e.}}
\newcommand\Ie{\foreign{I.e.}}
\newcommand\vs{\foreign{vs.~}}
\newcommand\cf{\foreign{cf.\@}}
\newcommand\resp{\foreign{resp.\xspace}}
\newcommand\etc{\foreign{etc.}}
\newcommand\etal{\foreign{et al}}

\newcommand\betti{\mathrm{b}}
\newcommand\Dih{\mathrm{Dih}}
\newcommand\eulercharacteristic{\chi}
\newcommand\fundamental{\homotopyGroup_1}
\newcommand\Grp{\mathsf{Grp}}
\newcommand\homologyGroup{\mathrm{H}}
\newcommand\homotopyGroup{\pi}
\newcommand\inj{\mathrm{i}}
\newcommand\poincare{\mathrm{P}}
\newcommand\PCat{\mathsf{Psh}}
\newcommand\restr{\rho}
\newcommand\SCat{\mathsf{Sh}}
\newcommand\sections{\mathrm{\Gamma}}
\newcommand\Sets{\textsf{Sets}}
\newcommand\ShfSpace{\mathrm{L}}
\newcommand\Shf{\mathrm{\Gamma}}
\newcommand\sphere{\mathbb{S}}
\newcommand\TopP{\mathsf{Top}_{\bullet}}
\newcommand\torus{\mathbb{T}}

\DeclareMathOperator\Coker{Coker}
\DeclareMathOperator\Hom{Hom}
\DeclareMathOperator\identity{id}
\DeclareMathOperator\Image{Im}
\DeclareMathOperator\Ker{Ker}
\DeclareMathOperator\Ob{Ob}
